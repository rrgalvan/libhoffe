\documentclass[11pt,article]{article}
% \usepackage[utf8]{inputenc}
\usepackage{a4wide}
\usepackage[T1]{fontenc}
\usepackage{tikz-uml}

\usepackage{helvet} % Use "Helvetica" as LaTeX font

\begin{document}

% LaTeX font for UML diagrams
% \tikzumlset{font=\sffamily}
\tikzumlset{font=\ttfamily}

\newcommand{\NV}{NV}
\newcommand{\SD}{SD}
\newcommand{\abstractF}[1]{\textit{#1}} % Font for abstract functions

\section{UML Diagrams}

Following the UML modeling language, we represent a class with a box
which has 3 sections. From top to bottom, these sections display: the
class name, the the instance variables and the methods of that class.

Abstract functions (whose definition is deferred for derived classes)
are denoted by \textit{italic font}.

The visibility of class members (variables, methods and constructors)
is specified by the following notation:
\begin{verbatim}
    +    Public
    -    Private
\end{verbatim}

\section{Mesh Data Types}

\begin{tikzpicture}
  % Draw the mesh class as a interface with two template parameters
  \umlclass[y=5,type=interface,template={real, \NV, \SD}]{abstract\_mesh}{}{
    + nvertices() : integer \\
    + dim() : integer \\
    + \abstractF{ncells}() : integer \\
    + \abstractF{vertices\_of\_cell}(int) : int[\NV] \\
    + \abstractF{coordinates\_of\_vertex}(int) : real[\SD]
  }
  % Attach a note to mesh class (information about template parameters)
  \umlnote[x=8,y=5]{abstract\_mesh}{
    \begin{description}
    \item[\sf real:] floating point type
    \item[\sf\NV:] nb. of vertices in each cell
    \item[\sf\SD:] space dimension
    \end{description}
    }
  % Draw the mesh1d class
  \umlclass[x=-3]{mesh1d}{}{
    + x(int) : real \\
    }
  % Define mesh1d as a specialization of mesh
  \umlreal[stereo={\NV $\rightarrow$ 2, \SD $\rightarrow$ 1}]{mesh1d}{abstract\_mesh}

  % Draw the mesh2d class
  \umlclass[x=3]{triang\_mesh2d}{}{
    + x(int) : real \\
    + y(int) : real
  }
  % Define mesh2d as a specialization of mesh
  \umlreal[stereo={\NV $\rightarrow$ 3, \SD $\rightarrow$ 2}]{triang\_mesh2d}{abstract\_mesh}
\end{tikzpicture}

\section{Quadrature Formula Data Types}

(...)

\section{Finite Element Data Types}

\begin{tikzpicture}
  % Draw the abstract finite element class as a interface
  \umlclass[type=interface]{abstract\_finite\_element}
  {
    - order\_: int \\
    - mesh\_ : abstract\_mesh \\
    - quad\_rule\_ : abstract\_quad\_rule \\
  }
  {
    + get\_order() : abstract\_mesh \\
    + get\_mesh() : abstract\_mesh \\
    + get\_quad\_rule() : abstract\_quad\_rule \\
    + get\_ndofs() : integer \\
    + get\_phi(int) : float[NQ] \\
    + get\_dx\_phi(int) : float[NQ] \\
    + \abstractF{reinit}(icell : int, order: int) \\
    (...)
  }
  % Attach a note (information about reinit)
  \umlnote[x=7]{abstract\_finite\_element}{ reinit() is responsible
    for associating a finite element object to the cell ``icell'' and
    resetting the corresponding data members.

    NQ is the nb. of quadrature nodes
    }
\end{tikzpicture}

\end{document}                  %
